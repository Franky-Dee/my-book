\chapter{Introduction}
\par{The demand for efficient and adaptable enterprise solutions has never been greater in today's
rapidly evolving technological landscape. Within the South African context, where the
intersection of academia and industry holds significant promise for innovation and economic
growth, the exploration of approaches to software development becomes imperative. This
introduction outlines the rationale, objectives, and structure of the research study aimed at
investigating the feasibility and efficacy of leveraging university students to develop a
commercial-grade software solution in the form of an Enterprise Resource Planning (ERP)
system in collaboration with TaskFlow a partnering North-West University-industry software
development company.}

\par{Collaboration between industry and academia is essential to benefit the global technological
village \citep{baig2018bridging}. Unfortunately, universities, especially in developing nations, 
often lack the resources to allow for the proper development of skills needed by students entering the software development workforce. Through collaborative initiatives, software companies possess
the necessary skills, know-how, and technology to improve the skills of university students in a
controlled environment that simulates real-world experience. For instance, companies
specialising in software development, like TaskFlow, integrate solutions such as Contact Centre,
Customer Relations Management, Dialer, and Helpdesk into a comprehensive Enterprise
Resource Planning (ERP) suite. This empowers their clients to optimise time and make
informed business decisions through tailor-made and integrated software that more efficiently
and effectively streamlines their entire business processes.}

\par{Integrated ERP systems significantly affect the organisation's value chain in both primary and secondary operations
 \citep{amini2020erp}. By connecting business and management
activities, enterprise resource planning (ERP) assists organisations in realising their full potential 
\citep{uccakturk2013effects}. Additionally, while ERP systems are pivotal to operational
success, the associated expenses of non-integrated software solutions present significant barriers
to cost and scalability. Consequently, many software-developing companies aim to minimise the
cost of training interns on the theory behind and use of integrated software tools, project management, 
and software development, especially when developing complex systems such as ERP suits. 
By developing an industry-approved software system, this research investigates how the North-West University in South Africa 
utilized computer science and information systems graduate students to address challenges in the South African IT 
industry through collaborative university/industry-applied project management and systems development methodologies. 
By doing so, the study explores the viability of creating a mutually beneficial
relationship between academia and industry, which can bring positive outcomes for both parties and students. 
As such, the research will delve into the potential of the curricula in Project management and System development at 
Honor's level to mitigate the challenges faced by the
industry while nurturing a dynamic and symbiotic partnership between academia and industry.}

\section{Problem Statement}
\par{The intersection of academia and industry presents unique opportunities and formidable challenges. By the late 
1990s, \cite{foley1997technology} argued that it is beneficial for graduate students to comprehend how the industry operates 
and how ideas and innovations are turned into the products and services society requires. As employers found it challenging to 
locate people with the knowledge and abilities needed for open positions, President Barack Obama emphasised the need for 
lawmakers and national higher education associations to urge educational institutions to not only set ambitious educational 
attainment targets but also ensure that their goals address the current gap in required industry skills \citep{barnett2011partnering}. 
Efficient learning and skill development in an educational setting have become increasingly important to ensure that 
individuals acquire the necessary expertise for success in various industries \citep{baig2018bridging}. }
\par{The central inquiry for this study emerges from the argument that tertiary education students
need to be trained for the requirements of a professional workspace \citep{baig2018bridging}. The
question arises: \textit{To what extent can postgraduate students be effectively taught to
develop industry-standard software solutions such as ERP systems amidst the multidimensional challenges of 
technical proficiency, client satisfaction, and academic rigour?}}

\section{Research Aims and Objectives}
\par{By developing an industry-approved software system, this research investigates how the North-West University in South 
Africa utilized computer science and information systems post-graduate students to address challenges in the South African 
IT INDUSTRY through collaborative university/industry-applied project management and systems development methodologies.
To answer the aim of the study, the first objective would be to define the essential components and prerequisites of a 
fully functional, versatile, and industry-approved software system. Secondly to understand the challenges the IT industry 
faces, especially in developing nations.  Thirdly, to investigate the critical project management and system development 
competencies and capabilities needed to create industry-standard software solutions and to what extent Higher Education 
Institutions (HEIs), especially in developing nations, address these requirements in their post-graduate computer science 
and Information curricula. And finally, how the industry can collaborate to transfer the necessary skills to HEIs learning 
processes to enable students to develop industry-approved software systems. }
\par{This study aims to evaluate the feasibility of creating a commercially viable software solution in
collaboration with a software development company as part of the NWU computer science and information systems department's 
prescribed curriculum on IT project management and software development honours program. Through collaboration with the 
industry, this study aims to showcase the technical abilities of computer science students by developing a software 
solution that meets commercial standards. This initiative also seeks to equip students with the necessary skills to become 
valuable contributors to the South African workforce. The project aims to conduct a pilot study to identify the key success 
factors in guiding postgraduate students at a Higher Education Institution (HEI) in an underdeveloped country. The goal is 
to provide these students with industry exposure in a formalised academic and learning environment, enabling them to 
develop economically viable software. The study seeks to determine how the students can be effectively mentored and trained 
and what support and resources are necessary to ensure their success.
Ultimately, the findings of this study will be used to inform the design of future programs that seek to foster innovation 
and economic growth in underdeveloped regions of the world.}
\par{This study explores the possibility of integrating projects and software development techniques taught in an 
educational setting by teaching existing structures and practices of a typical software development company in the 
South African industry. The objective is, therefore, to teach computer science and information systems graduates to 
identify market opportunities and, by applying industry practices and sound PM and SD methods, to develop software 
products that are commercially viable and aligned with market demands. As such, the research seeks to uncover optimal 
practices and tactics that improve educational insights regarding the teaching of PM and software development procedures, 
integrating cutting-edge best practices from the industry. To assess the software solution's viability as a marketable 
product, the research will conclude through an extensive evaluation by proficient experts and professionals in the software 
development industry. It will also involve thoroughly examining the client's satisfaction and feedback after receiving the 
final system, including any concerns they may raise. This
assessment will determine the critical success factors in developing commercial software, the roles of industry and 
academia in the learning process, and any necessary improvements to enhance the learning experience.}

\section{Hypothesis}
\par{Postgraduate computer science and information systems students can build a sophisticated software solution that is 
commercially viable, such as an ERP system when partnering with the expertise and resources of industry partners.}

\section{Methods of Investigation}
\par{Throughout the commencement of the research, the study will draw upon two main facets to substantiate the claims made in the paper's conclusions: the vast amount of academic literacy available and the other experiment based data collection processes referring to the development project as the main experiment. Both of these will be outlined below to provide clarity on both matters.}
\subsection{Literature Study}
\par{To enhance the credibility, accuracy, and validity of the study, a thorough and systematic review of pertinent 
literature will be conducted. This will involve targeting keywords such as higher education, academia, ERP systems, 
software success, project management, system implementation, and commercialisation. Research will be prioritised from a 
positivist perspective and examine peer-reviewed articles from reliable sources such as the NWU Library, Google Scholar, 
IEEE Xplore, and ACM Digital Library. In addition, grey literature from trusted sources like white papers and industry 
reports will be considered, as this research incorporates best practices from the industry. Duplicate articles or those 
not meeting our eligibility criteria will not be accepted. To locate important material, references from key papers will 
be scrutinised and analysed for citation patterns, thus seeking advice and insights from
subject matter experts. The focus will be on resources published between 2015 and 2024 to ensure relevance. 
Historical research to be focused on include:}
\par{\begin{itemize}
    \item Njanka, S. Q., Sandula, G., and Colomo-Palacios, R. (2021). It-business alignment: A
systematic literature review. Procedia Computer Science, 181:333–340.
    \item Kenge, R. and Khan, Z. (2020). A research study on the erp system implementation and
current trends in erp. Shanlax International Journal of Management, 8(2):34–39.
    \item Guo Chao Alex, P. and Chirag, G. o. C. I. S. (2014). Cloud erp: a new dilemma to modern
organisations? Journal of Computer Information Systems, 54(4):22–30.
    \item Bagchi-Sen, S., Baines, N., and Smith, H. L. (2022). Characteristics and outputs of univer-
sity spin-offs in the united kingdom. International Regional Science Review, 45(6):606–
635.
\end{itemize}}
\subsection{Methods of Investigation}
\par{As mentioned earlier, this study aims to thoroughly examine the literature on software systems and student development 
of software in developed nations to identify applicable lessons for developing environments like South Africa. A key 
objective is to equip students with the knowledge needed in PM and system development to create software that adheres to 
industry best practices. To achieve this, the study will assess the business case for higher educational institutions to 
develop commercially viable software for the industry as potential clients. Once
the need is established, the next step will be identifying the technological and functional requirements for 
creating software that real-world clients can use to support their daily operations. Once the software is completed, 
the study will evaluate the usability of the software as well as the learning experience and the proficiency of postgraduate 
students in applying project management and systems development methodologies to complex software development. The 
aim is to offer significant and insightful knowledge that can be utilised by higher education institutions (HEIs) in 
underdeveloped countries to improve the functionality, efficiency, and acceptance of their software and systems 
development projects. This will help HEIs teach the most relevant and useful theories of project management and teamwork, 
ultimately enhancing their students employability and marketability in the industry.}
\par{As such, the researcher will serve as an integral part of the project, overseeing software development. As 
mentioned, according to set project management milestones, the software system will undergo testing by reviewers with 
experience in the information technology and ERP industries. All lessons learned and the input from reviewers will be 
compiled to produce a workable framework that can direct HEIs in leveraging project management and software development 
curricula to develop commercially viable software solutions and systems for internal and industry applications.}

\section{Chapter Division}
\par{This thesis is structured into six main chapters, each meticulously crafted to delve into specific
facets of the research endeavour, elucidating methodologies, literature insights, empirical findings, and 
concluding reflections. The chapters are as follows:}
\par{The Research Design chapter serves as the foundational cornerstone of the thesis, this chapter
provides a detailed exposition of the research methodology and paradigm guiding the study. Adopting a design science 
research methodology and a positivist paradigm, this section navigates the theoretical underpinnings of the research 
approach, outlining its epistemological and ontological foundations. Through meticulous delineation of research
methods, including data collection techniques, sampling strategies, and analytical frameworks, this chapter lays the 
groundwork for subsequent empirical inquiry. Once this chapter has concluded, a review of the current literature in the 
field will be done to pinpoint previous scientific findings that will benefit the current study.}
\par{The Literature Study chapter embarks on an immersive journey through the annals of academic literature, traversing 
diverse domains ranging from ERP systems to implementation frameworks. The section unfolds with a lucid articulation of 
the problem statement, contextualising the research within the broader academic discourse. Subsequent sections delve into 
the intricacies of ERP systems, exploring their architectural nuances, functional capabilities, and strategic significance 
within organisational contexts. Furthermore, the chapter probes into the value chain framework, dissecting the 
interconnected dynamics of people, processes, and technology in driving organisational success. Anchored by 
comprehensive reviews of commercialization strategies, software implementation frameworks, and the role of students in 
industry, this chapter serves as a beacon of scholarly inquiry, illuminating the theoretical landscape underpinning the 
research endeavor. Once the theoretical landscape has been illuminated, a case study can be conducted to generate findings 
regarding the research question, and help the research endeavour edge closer to final conclusion.}
\par{The Case Study embarks on a voyage of empirical exploration, this chapter unfurls the intricacies of the case study 
conducted within the hallowed confines of TaskFlow. With a keen focus on project design and execution, this section 
unveils the methodological blueprint guiding the project life cycle. Through an exhaustive examination of project 
management methodologies, agile development frameworks, and strategic planning processes, this chapter offers 
unparalleled insights into the operational dynamics underpinning software development endeavours. Furthermore, by 
delineating the project pipeline and process, this section charts the trajectory of the project's evolution, 
elucidating key milestones, challenges encountered, and lessons learned along the way. Once the case study has 
concluded, and sufficient data is collected around the subject at hand, an expert reviewer will reveal the depth of 
the truth within the findings and help the project reach its final conclusion.}
\par{In pursuit of academic rigour and scholarly validation, the Expert Reviewer chapter invites the discerning gaze 
of domain experts to scrutinise the research methodology and findings. Through structured interviews, expert 
assessments, and critical appraisals, this section seeks to engender a dialectic discourse, fostering intellectual 
exchange and epistemic enrichment. By subjecting the research endeavour to rigorous scrutiny, this chapter 
endeavours to enhance the robustness and validity of the research findings, ensuring their salience and applicability 
within the broader academic community. The chapter that follows will discuss the results that were found throughout the study.}
\par{The Results and Discussion serves as the crucible of empirical inquiry, this chapter presents and analyses the 
empirical findings gleaned from the case study and expert review. Through meticulous data analysis techniques, 
including qualitative coding, thematic analysis, and comparative synthesis, this section unveils the rich tapestry of 
insights garnered from the research endeavour. Furthermore, by fostering a dialectic discourse between empirical findings 
and theoretical frameworks, this chapter endeavours to unravel the underlying mechanisms and causal relationships shaping 
the research phenomena. The chapter that follows will make the final conclusions regarding the total study and its 
findings; it will act as a summary for the research project and summarise the findings to draw the final conclusion.}
\par{The conclusions chapter aims to synthesise key insights, implications, and avenues for future research, this 
chapter offers a reflective denouement to the research endeavour. Through a judicious synthesis of empirical findings and 
theoretical frameworks, this section distils the essence of the research inquiry, elucidating its broader implications and 
relevance within academic and practical contexts. Furthermore, by delineating avenues for future research and scholarly 
inquiry, this chapter seeks to catalyse intellectual curiosity and foster a culture of continuous learning and innovation.}
\par{In summation, this research endeavour represents a concerted effort to bridge the gap between academic theory and 
industrial practice, laying the foundation for the subsequent chapter, "Research Design." By elucidating the theoretical 
underpinnings and methodological frameworks guiding the study, the forthcoming chapter endeavors to translate lofty 
aspirations into tangible research methodologies, fostering innovation, collaboration, and economic prosperity within 
the South African software development landscape.}