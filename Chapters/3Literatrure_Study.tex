\chapter{Literature Study}
\par{Enterprise Resource Planning (ERP) systems have become a cornerstone in the modern business
landscape, facilitating the integration and management of various organizational processes. 
These systems, characterized by their comprehensive features and functionalities, are designed 
to streamline operations, enhance efficiency, and provide real-time insights. This literature 
review explores the multifaceted nature of ERP systems, delving into their features, the 
competitive dynamics within the ERP industry, and the essential components that define an 
industry-approved software system.

The subsequent sections address the broader context of the information technology industry, 
highlighting the challenges it faces globally and within developing nations specifically. The role 
of universities in bridging the gap between academic knowledge and industry requirements is 
examined, with a comparative analysis of the unique challenges faced by developing nations versus 
more developed regions.

A thorough examination of the value chain in software development is provided, focusing on 
the critical aspects of people, processes, technology, hardware, software, and infrastructure. 
This section also considers the economic viability of partnerships between academia and industry, 
emphasizing the mutual benefits and resource-sharing opportunities.

The review further investigates the commercialization of software, discussing various methods 
and offering recommendations for effective commercialization strategies. The role of students in the
industry is explored, considering the advantages, potential threats, and comparisons with similar 
projects undertaken elsewhere. An in-depth analysis of project management methodologies relevant to software development is presented, along with specific recommendations tailored to the context of ERP systems. Implementation frameworks are examined to identify critical competencies required for successful 
software development, followed by targeted recommendations. Finally, the review discusses the artefact of ERP systems, presenting methods for measuring the success and functionality of such systems. The Technology Acceptance Model (TAM) is utilized as a framework to evaluate user acceptance and effectiveness.

This comprehensive literature review aims to provide a detailed understanding of the 
complexities involved in students building ERP systems for industry, offering insights into best 
practices, challenges, and strategic approaches for successful implementation and commercialization.}

\section{ERP Systems}
\par{The ERP archive, which dates back to possibly 1970, was started with the intention of integrating business activities \citep{shields2004business}. ERP was first used at the beginning of 1990, and it was named by the Gartner Group \citep{chang2000delphi}. The early 1990s saw the introduction of ERP by software companies like SAP. In 1992, SAP released the R/3 version once more. Customer-server hardware structure was added to the SAP R/3 so that it could operate on many stages at once \citep{jacobs2007enterprise}. By 2000, all the main ERP software system providers had solved the Y2K challenge. By connecting business and management activities, enterprise resource planning (ERP) tools assist organisations in realising their full potential \citep{uccakturk2013effects}. Business patterns will shift over the next ten years as a result of modifications to a vertical market, application techniques, and the ERP cost structure. Cloud application models are stored in a lot of data. SaaS, for instance, is attracting businesses' attention. Business enterprises seeking to reduce significant capital costs through a monthly subscription model have embraced the ERP pricing model, which charges based on usage \citep{kenge2020research}.}
\subsection{Features of ERP Systems}
\subsection{Competition within the ERP System Industry}
\subsection{Components of an Industry Approved Software System}
\subsection{Components of an Industry Approved ERP System}

\section{The Information Technology Industry}
\subsection{Challenges within the Information Technology Industry}
\subsection{Challenges Exclusive to Developing Nations}
\subsubsection{The Purpose of a University}
\subsection{Developing Nations vs. The Rest of the World}

\section{The Value Chain}
\subsection{People}
\subsection{Processes}
\subsection{Technology}
\subsubsection{Hardware}
\subsubsection{Software}
\subsubsection{Infrastructure}
\subsection{The Economic Viability of Academic and Industry Partnership}
\subsection{Industry Offering Resources to Universities}

\section{Commercialization of Software}
\subsection{Methods of Commercialization}
\subsubsection{Recommendation}

\section{Students in Industry}
\subsection{Advantages}
\subsection{Threats}
\subsection{Similar Projects}

\section{Project Management Methodologies}
\subsection{Software Development Management Methodologies}
\subsection{Recommendations}

\section{Implementation of ERP Systems}
\par{The seven main steps in the ERP deployment process include business process analysis, software installation, data migration, software performance testing, user training, complete deployment, and post-implementation support \citep{ly2020definitive}. These main key steps are examined in greater depth in the sections that follow.}
\subsubsection{Business Process Research and Requirements Gathering}
\par{The first step in the ERP implementation procedure is to define the needs, goals, and scope of the ERP within the specified business' process. Building a team that can work on the ERP implementation project from beginning to end is also necessary \citep{wetherbe2006information}. Within the team there would mainly be two roles, software developers and the sub roles that preside within this overarching role, as well as business analysts that are responsible for documenting and understanding the business process and how it will be converted into an ERP system. This step within the implementation process is largely the responsibility of the business analysts. The responsibilities of the business analyst within this step of the process would be as follows \citep{yusuf2004enterprise}:}
\begin{itemize}
    \item Analyse, record, and describe an organisation's current procedures. 
    \item Look for significant difficulties, process waste, and customer-focused problems.
    \item Establish clear objectives for an ERP deployment that are connected to the major success areas, and quantify them precisely.
    \item Create a cost budget and a solid timetable.
\end{itemize}
\subsubsection{Software Installation}
\par{Following the first step's creation of new procedure flows, the team should have a new business process plan in place. The architecture and infrastructure for software, such as the data store, data presentation, and internet accessibility, will be installed and constructed by the software developers who have recieved the business requirements from the business analyst.}
\subsubsection{Data Migration}
\par{In this stage, all data is transferred to a new software platform. Before the data is transferred to a new site, it should all be reviewed and adjusted to ensure a smooth mapping process. Data mapping between the previous and new store locations, data transfer, and the configuration of a new data storage location are all included in this stage.}
\subsubsection{Testing}
\par{All data interfaces, functionality, and real-time data transactions are tested by the quality engineer. Users need to make sure that information is accurately moving between various departments and that the system is functionally sound. The testing process is important to ensure that no edge cases arise during the use of the deployed system and that the system behaves as predicted.}
\subsubsection{User Training}
\par{Personnel react to change management, and user training relies on the intricate nature of the ERP program. Up to 56 percent of ERP deployment cases after go live result in production halts under training.}
\subsubsection{Final Deployment}
\par{Depending on the scale of the ERP software and the assets available, the organisation might select one of the following three ways:}
\begin{itemize}
    \item Big-Bang Method: A one-day switch from the outdated to the updated software. This method is quick and inexpensive, but any inefficiency in deployment could lead to a serious issue during use.
    \item Phased Approach: A longer-term, function- or unit-specific phased transition.
    \item Parallel Operation Approach: In order to reduce risk, users use both new and old systems simultaneously.
    This method has greater operational expenses for the two platforms and necessitates more time for repetition of work.
\end{itemize}
\subsubsection{Support}
\par{Performance review of ERP projects is crucial and should be done for the duration of the project. Key performance indicators that can be compared are as follows:}
\begin{itemize}
    \item Real implementation costs compared to the budgeted amount
    \item The return on investment for the project
    \item The assessment of mistakes or human errors
    \item The efficiency of the supply chain and production
    \item The satisfaction of customers and their willingness to continue working with the development company
\end{itemize}
\subsection{ERP Implementation Time}
\par{Once the software system is launched, an ERP system implementation project may take anywhere from three months to several years to complete and fully implement to a point where both client and development company are satisfied \citep{sankar2006implementation}. The size of the organisation, the volume of data, the number of users, and the resources all affect how long the ultimate project implementation takes \citep{pelphrey2015directing}.}
\subsection{Critical Competencies for Software Development}

\section{Artefact}
\subsection{Methods for Measuring Artefact Success}
\subsection{Methods for Measuring Artefact Functionality}
\subsection{TAM Model}