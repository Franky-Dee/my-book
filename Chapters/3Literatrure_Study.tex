\chapter{Literature Study}
\par{Enterprise Resource Planning (ERP) systems have become a cornerstone in the modern business 
landscape, facilitating the integration and management of various organizational processes. 
These systems, characterized by their comprehensive features and functionalities, are designed 
to streamline operations, enhance efficiency, and provide real-time insights. This literature 
review explores the multifaceted nature of ERP systems, delving into their features, the 
competitive dynamics within the ERP industry, and the essential components that define an 
industry-approved software system.}
\par{The subsequent sections address the broader context of the information technology industry, 
highlighting the challenges it faces globally and within developing nations specifically. The role 
of universities in bridging the gap between academic knowledge and industry requirements is 
examined, with a comparative analysis of the unique challenges faced by developing nations versus 
more developed regions.}
\par{A thorough examination of the value chain in software development is provided, focusing on 
the critical aspects of people, processes, technology, hardware, software, and infrastructure. 
This section also considers the economic viability of partnerships between academia and industry, 
emphasizing the mutual benefits and resource-sharing opportunities.}
\par{The review further investigates the commercialization of software, discussing various methods 
and offering recommendations for effective commercialization strategies. The role of students in the
industry is explored, considering the advantages, potential threats, and comparisons with similar 
projects undertaken elsewhere.}
\par{An in-depth analysis of project management methodologies relevant to software development 
is presented, along with specific recommendations tailored to the context of ERP systems. 
Implementation frameworks are examined to identify critical competencies required for successful 
software development, followed by targeted recommendations.}
\par{Finally, the review discusses the artefact of ERP systems, presenting methods for
measuring the success and functionality of such systems. The Technology Acceptance Model (TAM)
is utilized as a framework to evaluate user acceptance and effectiveness.}
\par{This comprehensive literature review aims to provide a detailed understanding of the 
complexities involved in students building ERP systems for industry, offering insights into best 
practices, challenges, and strategic approaches for successful implementation and commercialization.}
%====================================================================================================
\section{ERP Systems}
\par{The ERP archive discovered in the distant past may date back to 1970, when business process 
integration was the primary goal. The ERP was first used by the Gartner Group, and it went into 
use at the beginning of 1990. The early 1990s saw the introduction of ERP by software companies 
like SAP. In 1992, SAP released the R/3 version once more. Customer-server hardware structure was
 added to the SAP R/3 so that it could operate on many stages at once. By 2000, all of the main 
 ERP software system providers had solved the Y2K challenge. By connecting business and management 
 activities, enterprise resource planning (ERP) tools assist organisations in realising their full 
 potential \citep{uccakturk2013effects}.}

\subsection{Features of ERP Systems}
\subsection{Competition within the ERP System Industry}
\subsection{Components of an Industry Approved Software System}
\subsection{Components of an Industry Approved ERP System}
%====================================================================================================
\section{The Information Technology Industry}
\subsection{Challenges within the Information Technology Industry}
\subsection{Challenges Exclusive to Developing Nations}
% What caused this? Is there a politicol drive?
\subsubsection{The Purpose of a University}
% What does the government want from the university
\subsection{Developing Nations vs. The Rest of the World}
%====================================================================================================
\section{The Value Chain}
\subsection{People}
\subsection{Processes}
\subsection{Technology}
\subsubsection{Hardware}
\subsubsection{Software}
\subsubsection{Infrastructure}
\subsection{The Economic Viability of Academic and Industry Partnership}
\subsection{Industry Offering Resources to Universities}
%====================================================================================================
\section{Commercialization of Software}
\subsection{Methods of Commercialization}
\subsubsection{Recommendation}
%====================================================================================================
\section{Students in Industry}
\subsection{Advantages}
\subsection{Threats}
\subsection{Similar Projects}
%====================================================================================================
\section{Project Management Methodologies}
\subsection{Software Development Management Methodologies}
\subsection{Recommendations}
%====================================================================================================
\section{Implementation Frameworks}
\subsection{Critical Competencies for Software Development}
\subsection{Recommendation}
%====================================================================================================
\section{Artefact}
\subsection{Methods for Measuring Artefact Success}
\subsection{Methods for Measuring Artefact Functionality}
\subsection{TAM Model}